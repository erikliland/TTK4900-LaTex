%!TEX root = ../TTK4900-MHT.tex

\section*{Problem description}
% \addcontentsline{toc}{chapter}{Problem description}

\subsection*{Background}\label{sub:background}
Multitarget tracking is a key ingredient in collision avidance system for autonomous vehicles. Multi-frame tracking methods are commonly acknowledged as gold standards for multi-target tracking. The purpose of this master thesis is to develop a complete multi-frame system for autonomous ships, based on sensor inputs from radar and the \gls{ais}.

\subsection*{Proposed tasks}\label{sub:proposed_tasks}
The following task are proposed for this thesis:
\begin{itemize}
	\item{Extend an integer-linear-programming (ILP) based tracking method with suitable algorithms for track initiation and track management}
	\item{Develop a framework for fusion between radar tracks and AIS tracks}
	\item{Develop alternatives to N-scan pruning in order to enhance the computational efficiency of the tracking method}
	\item{Implement the tracking system in Python and/or C++}
	\item{Test the tracking system on simulated data}
	\item{Test the tracking system with real data recorded with the Navico 4G broadband radar mounted on Telemetron}
\end{itemize}

\subsection*{Autosea}\label{sub:autosea}
This thesis is associated with the AUTOSEA project, which is collaborative research project between NTNU, DNV GL, Kongsberg Maritime and Maritime Robitics, focused on achieving world-leading competence and knowledge in the design and verification of methods and systems for sensor fusion and collision avoid- ance for ASVs. The project has access to supervision and physical test platforms through our industry partners.