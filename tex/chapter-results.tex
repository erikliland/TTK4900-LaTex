%!TEX root = ../TTK4900-MHT.tex

\chapter{Results}\label{chapter:results}
\section{Testing scheme}
The performance evaluation of the MHT tracking system is tedious in that it is necessary to test very many different situations to get a good understanding of how the system is performing. The two largest factors contributing to the difficulty is the random nature of the clutter and lost detections. It is also desirable to evaluate the initialization and tracking performance under both varying environmental (external) conditions and tuning (internal) setting. We want good tracking of targets with low probability of detections in cluttered environment, and secondly it must be able to do this within the time frame of the radar rotation period. The initialization module must be able to detect targets with probability of detection lower than unity without initializing too may false tracks into the MHT algorithm. The testing is separated into two parts; initialization and tracking.

The performance metrics for the initialization module is how long time it takes to initialize the correct tracks, which is tested under a range of internal and external conditions, see (\ref{eq:init_test_table}). All combinations of these parameters were simulated on all scenarios, see Table~\ref{tab:ais_scenarios}, which are the same routes but with different \gls{ais} configurations. From these simulations, the time to initiate true targets and amount of false targets are calculated. A track is categorized as correct initialized if the state difference between the true track and the initial track is less that a threshold. All initial tracks that does not correspond to a true track is categorized as erroneous. To analyse the impact of the erroneous tracks, the lifespan of falsely initiated tracks is plotted to see whether they die out at the same rate as they are initiated, or if they accumulate.  
\begin{equation}\label{eq:init_test_table}
\begin{split}
\V{P_D} &= \begin{bmatrix} 1.0 & 0.8 & 0.6 \end{bmatrix} \\
\V{M/N} &= \begin{bmatrix} 	(1/1) & (1/2) & (1/3) & (1/4) \\
							(2/2) & (2/3) & (2/4) & (2/5) \\
							(3/3) & (3/4) & (3/5) & (3/6)
		   \end{bmatrix} \\
\V{\lambda_\phi} &= \begin{bmatrix} 0 & 2\cdot10^{-6} & 4\cdot10^{-6} \end{bmatrix}
\end{split}
\end{equation}

When testing the tracking performance, it is desirable to remove the variable of initialization to better see difference in \emph{tacking} rather than \emph{initialization}. Therefore are all the simulations testing tracking performance carried out with all targets correctly initialized at initial time, and with the initiator set to \(M=2, N=4\) such that the unused measurements from the tracking algorithm would be treated as normal. This would also give lost targets a change to get re-initialized, which is an important property for any safety critical system.
\begin{equation}\label{eq:tracking_test_table}
\begin{split}
\V{P_D} &= \begin{bmatrix} 1.0 & 0.8 & 0.6 \end{bmatrix} \\
\V{N} &= \begin{bmatrix} 1 & 3 & 6 & 9 \end{bmatrix} \\
\V{\lambda_\phi} &= \begin{bmatrix} 0 & 2\cdot10^{-6} & 4\cdot10^{-6} \end{bmatrix}
\end{split}
\end{equation}
Both initialization- and tracking performance is averaged over a set of 50 Monte Carlo simulations with differently seeded clutter- and detections points.

\section{Scenario}\label{sec:scenario}
All simulations in this work is based on a generated scenario, as shown in Figure~\ref{fig:test_scenario}, with black dots marking the initial time and position. The radar range is 5500 meter (~3 \glspl{nm_acr}), which gives an area of surveillance of approximately 95 square km. The scenario contains 16 targets, which all starts inside the observable area of the radar. The scenario contains a mixture of fast and slow moving vessels, some with sharp turns and some almost at stand still. 
\begin{figure}[H]
\centering
\includegraphics[width = .8\textwidth]{Figures/scenarioTruth.pdf}
\caption{True tracks}\label{fig:test_scenario}
\end{figure}
From this base scenario, five scenarios where generated with different AIS configuration on the vessels, see Table~\ref{tab:ais_scenarios}. The first scenario represent the baseline with only radar information available, whereas the rest have some level of AIS information. Scenario 2 adds three class B AIS transmitters, and is representing a situation where all the targets are smaller vessels with some voluntarily installed AIS transceivers. In scenario 3, all vessels have AIS class B installed. This scenario represents a best case situation regarding yacht and leisure vessels from an autonomous anti collision perspective and is only realistic if AIS class B where to be mandatory for these vessel classes. Scenario 4 is the same as scenario 2, with the difference that the vessels have class A transmitters in stead of class B. This gives them higher and smarter rate of transmission, which in theory should improve tracking under challenging conditions. This scenario can be viewed as a few commercial vessels travelling in between a large group of yachts. The last scenario, where all targets are equipped with class A transmitters is the ultimate situation for any fusion tracking system. This case would be realistic in a crowded professional working area, for instance harbours, fishing areas and off-shore installations. 
\begin{table}
\centering
	\begin{tabularx}{0.5\textwidth}{XXXXXX}
	  \multicolumn{5}{c}{AIS scenario configuration} \\
	  \toprule
	  		 & \multicolumn{5}{c}{Scenario} \\
	  Target & 0 	& 1 	&  2 	&  3	& 4  	\\
	  \midrule
	  0 	& --- 	& --- 	& B 	& --- 	& A 	\\
	  1 	& --- 	& --- 	& B 	& --- 	& A 	\\
	  2 	& --- 	& --- 	& B 	& --- 	& A 	\\
	  3 	& ---	& B 	& B 	& A 	& A 	\\
	  4 	& --- 	& --- 	& B 	& --- 	& A 	\\
	  5 	& --- 	& --- 	& B 	& --- 	& A 	\\
	  6 	& --- 	& --- 	& B 	& --- 	& A 	\\
	  6 	& --- 	& --- 	& B 	& --- 	& A 	\\
	  6 	& --- 	& --- 	& B 	& --- 	& A 	\\
	  7 	& --- 	& --- 	& B 	& --- 	& A 	\\
	  8 	& --- 	& --- 	& B 	& --- 	& A 	\\
	  9 	& ---	& B 	& B 	& A 	& A 	\\
	  10 	& --- 	& --- 	& B 	& --- 	& A 	\\
	  11 	& --- 	& --- 	& B 	& --- 	& A 	\\
	  12 	& ---	& B 	& B 	& A 	& A 	\\
	  13 	& --- 	& --- 	& B 	& --- 	& A 	\\
	  \bottomrule
	\end{tabularx}~\caption{AIS scenario configuration}\label{tab:ais_scenarios}
\end{table}

\section{Simulation}
For each simulation