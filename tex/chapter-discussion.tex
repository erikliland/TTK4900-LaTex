%!TEX root = ../TTK4900-MHT.tex

\chapter{Discussion}\label{chapter:discussion}
The 2/2\&m/n initialization logic have proved to be a simple and reliable way to incorporate initialization to \gls{tomht}. Its performance is tuning dependent and the simulation results indicates that there are multiple alternatives which all gives satisfactory results, though with a trade off between initialization time and amount of erroneous tracks.

The marginal increase in tracking percentage of medium to high \gls{Pd} and N variations when aiding the tracking system with \gls{ais} can indicate that the \gls{ais} aiding will have greatest impact in situations where a vessels is in a temporary radar shadow. This is situations where a target for a limited time have a reduced \gls{Pd} caused by blocking objects other vessels and land or masking phenomena like heavy snow or rain. Kalman smoothing of the outgoing track list have showed to provide a better estimate of the true tracks for all window sizes, and with a greater gain for larger window sizes. 



\section{Future work}\label{sec:future-work}
There are several unanswered questions that need to be addressed in order to achieve a full utilization of the \gls{ais} information. The maybe most important is what to do when a track previously associated with an MMSI no longer has new AIS measurements from that MMSI inside its gate? This is a natural situation since the radar update period is often smaller that the \gls{ais} update period, but for how long time shall that track continue to be associated with that \gls{mmsi}. And equally important is the decisions that needs to me made when the associated \gls{mmsi} is appearing outside the gates for that track. One option is to always make hypothesis where the associated \gls{mmsi} measurements is, but this could lead to large jumps in the target state and if the associated \gls{mmsi} was wrong, this would force a target of its radar track and lock it onto an \gls{ais} track.

An other open question is how to utilize the \gls{ais} meta-data to improve the tracking? From AIS length information, it might be possible to estimate the maximum turning rate for a vessel, which can be used to more accurately set the model parameters for a single target.

The core issue with any \gls{mht} algorithm is the exponential growth in the problem size. A novel approach to this could be to formulate a pruning score function for each node tree with the goal of removing nodes that does not add information while still having enough nodes to never render the original association optimization problem infeasible.